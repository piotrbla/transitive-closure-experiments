\documentclass[a4paper,12pt]{scrartcl}
\usepackage{polski}
\usepackage[T1]{fontenc}
\usepackage[utf8]{inputenc}
\usepackage{geometry}
\usepackage{tikz}
\usepackage{listings}
\usepackage{placeins}
\usetikzlibrary{plotmarks}
\title{Raport}
\subtitle{Wpływ liczby wątków na czas wykonywania - Optimal BST}
\author{Piotr Błaszyński}
\date{\today}

\begin{filecontents}{tnorm.data}
#numthreads 	time [s]
0   188.650966
1   184.661649
2   158.061008
3   118.205924
4   92.788404
5   74.941802
6   62.997307
7   55.677385
8   47.967482
9   43.426307
10  40.259185
11  35.993129
12  34.595824
13  32.19857
14  30.683281
15  29.147565
16  30.389329
17  29.545287
18  31.015962
19  30.544127
20  30.874198
21  29.219287
22  29.963998
23  29.697854
24  29.967067
25  29.370663
26  30.694492
27  29.897847
28  30.145456
29  30.015018
30  30.149217
31  29.366212
32  30.030107
\end{filecontents}


\begin{filecontents}{tif.data}
#numthreads 	time [s]
0   94.191369 
1   57.18226
2   34.87422
3   24.759321
4   18.385709
5   15.307084
6   12.752232
7   10.967784
8   9.978986
9   9.111519
10  8.236732
11  7.681049
12  7.07542
13  7.87217
14  8.292541
15  8.626039
16  8.776943
17  8.089999
18  7.933015
19  7.720661
20  7.590544
21  7.3329
22  7.084401
23  6.986372
24  6.908875
25  6.575586
26  6.729274
27  6.785697
28  6.7674
29  6.823531
30  6.764758
31  6.68795
32  7.05399
\end{filecontents}

\begin{filecontents}{s1.data}
#numthreads 	s
0    1.000000000
1    1.021603387
2    1.193532601
3    1.595951875
4    2.033130843
5    2.517299571
6    2.994587785
7    3.388287112
8    3.932892829
9    4.344163228
10   4.685911203
11   5.241304972
12   5.4529982
13   5.858985849
14   6.148330943
15   6.472271903
16   6.207802943
17   6.385145827
18   6.082383194
19   6.176341724
20   6.110311465
21   6.45638499
22   6.295921058
23   6.352343371
24   6.295276278
25   6.423108869
26   6.146085298
27   6.309851208
28   6.258023299
29   6.285219153
30   6.257242634
31   6.424082411
32   6.282061066
\end{filecontents}

\begin{filecontents}{s2.data}
#numthreads 	s
0   2.002847692
1   3.299116999
2   5.409467681
3   7.619391743
4   10.26073925
5   12.32442221
6   14.79356445
7   17.20046328
8   18.9048232
9   20.70466692
10  22.90361833
11  24.56057317
12  26.66286468
13  23.96429015
14  22.74947643
15  21.86994123
16  21.4939263
17  23.31903453
18  23.78048775
19  24.4345615
20  24.85341841
21  25.72665194
22  26.62906377
23  27.00270842
24  27.30559838
25  28.68960515
26  28.03437132
27  27.80126581
28  27.87643201
29  27.64711789
30  27.88731925
31  28.20759216
32  26.74386638
\end{filecontents}

\begin{document}
\maketitle
\lstset{language=C++,
                basicstyle=\ttfamily,
                keywordstyle=\color{blue}\ttfamily,
                stringstyle=\color{red}\ttfamily,
                commentstyle=\color{green}\ttfamily,
                frame=single, 
                breaklines=true,
                morecomment=[l][\color{magenta}]{\#}
}

Dla poniższego kodu wygenerowanego przez narzędzie dapt/stencil:
\begin{lstlisting}
for (int c0 = floord(-n + 1, 16) + 2; c0 <= floord(n - 3, 16) + 1; c0 += 1) {
\end{lstlisting}

Wygenerowanego na podstawie kodu sekwencyjnego obliczającego optymalne ułożenie BST:
\begin{lstlisting}
  for (t1=0;t1<=floord(44*n-69,475);t1++) {
\end{lstlisting}

Oraz dla poniższego kodu wygenerowanego przez narzędzie dapt/stencil:
\begin{lstlisting}
for (int c0 = floord(-n + 1, 16) + 2; c0 <= floord(n - 3, 16) + 1; c0 += 1) {
\end{lstlisting}

Wygenerowanego na podstawie ulepszonego (kod z if) kodu sekwencyjnego obliczającego optymalne ułożenie BST:
\begin{lstlisting}
  for (t1=0;t1<=floord(44*n-69,475);t1++) {
\end{lstlisting}


przyjmując wartości dla MACRO\_NUM\_THREADS z przedziału od 1 do 24 (oraz wykonując wersję sekwencyjną, bez blokowania) uzyskałem następujące wyniki (zero oznacza czas wykonania dla wersji sekwencyjnej):

\begin{tabular}{|r|l|l|}
num\_threads & normal obst & improved obst\\  \hline 
0  & 188.650966 & 94.191369  \\ \hline 
1  & 184.661649 & 57.18226   \\ \hline 
2  & 158.061008 & 34.87422   \\ \hline
3  & 118.205924 & 24.759321  \\ \hline
4  & 92.788404 & 18.385709  \\ \hline
5  & 74.941802 & 15.307084  \\ \hline
6  & 62.997307 & 12.752232  \\ \hline
7  & 55.677385 & 10.967784  \\ \hline
8  & 47.967482 & 9.978986   \\ \hline
9  & 43.426307 & 9.111519   \\ \hline
10 & 40.259185 & 8.236732   \\ \hline
11 & 35.993129 & 7.681049   \\ \hline
12 & 34.595824 & 7.07542    \\ \hline
13 & 32.19857 & 7.87217    \\ \hline
14 & 30.683281 & 8.292541   \\ \hline
15 & 29.147565 & 8.626039   \\ \hline
16 & 30.389329 & 8.776943   \\ \hline
17 & 29.545287 & 8.089999   \\ \hline
18 & 31.015962 & 7.933015   \\ \hline
19 & 30.544127 & 7.720661   \\ \hline
20 & 30.874198 & 7.590544   \\ \hline
21 & 29.219287 & 7.3329     \\ \hline
22 & 29.963998 & 7.084401   \\ \hline
23 & 29.697854 & 6.986372   \\ \hline
24 & 29.967067 & 6.908875   \\ \hline
25 & 29.370663 & 6.575586   \\ \hline
26 & 30.694492 & 6.729274   \\ \hline
27 & 29.897847 & 6.785697   \\ \hline
28 & 30.145456 & 6.7674     \\ \hline
29 & 30.015018 & 6.823531   \\ \hline
30 & 30.149217 & 6.764758   \\ \hline
31 & 29.366212 & 6.68795    \\ \hline
32 & 30.030107 & 7.05399    \\ \hline
\end{tabular} 

\newpage
Uzyskane przyśpieszenia:

\begin{tabular}{|r|l|l|}
num\_threads & speedup S1 & speedup S2 \\  \hline 
0  & 1.000000000 & 2.002847692  \\ \hline 
1  & 1.021603387 & 3.299116999  \\ \hline 
2  & 1.193532601 & 5.409467681  \\ \hline
3  & 1.595951875 & 7.619391743  \\ \hline
4  & 2.033130843 & 10.26073925  \\ \hline
5  & 2.517299571 & 12.32442221  \\ \hline
6  & 2.994587785 & 14.79356445  \\ \hline
7  & 3.388287112 & 17.20046328  \\ \hline
8  & 3.932892829 & 18.9048232  \\ \hline
9  & 4.344163228 & 20.70466692  \\ \hline
10 & 4.685911203 & 22.90361833  \\ \hline
11 & 5.241304972 & 24.56057317  \\ \hline
12 & 5.4529982 & 26.66286468  \\ \hline
13 & 5.858985849 & 23.96429015  \\ \hline
14 & 6.148330943 & 22.74947643  \\ \hline
15 & 6.472271903 & 21.86994123  \\ \hline
16 & 6.207802943 & 21.4939263  \\ \hline
17 & 6.385145827 & 23.31903453  \\ \hline
18 & 6.082383194 & 23.78048775  \\ \hline
19 & 6.176341724 & 24.4345615  \\ \hline
20 & 6.110311465 & 24.85341841  \\ \hline
21 & 6.45638499 & 25.72665194  \\ \hline
22 & 6.295921058 & 26.62906377  \\ \hline
23 & 6.352343371 & 27.00270842  \\ \hline
24 & 6.295276278 & 27.30559838  \\ \hline
25 & 6.423108869 & 28.68960515  \\ \hline
26 & 6.146085298 & 28.03437132  \\ \hline
27 & 6.309851208 & 27.80126581  \\ \hline
28 & 6.258023299 & 27.87643201  \\ \hline
29 & 6.285219153 & 27.64711789  \\ \hline
30 & 6.257242634 & 27.88731925  \\ \hline
31 & 6.424082411 & 28.20759216  \\ \hline
32 & 6.282061066 & 26.74386638  \\ \hline
\end{tabular} 

\begin{tikzpicture}[yscale=0.05,xscale=.4, font=\sffamily]
 	%axis
	\draw (0,0) -- coordinate (x axis mid) (33,0);
    	\draw (0,0) -- coordinate (y axis mid) (0,189.2);
    	%ticks
    	\foreach \x in {0,5,...,32}
     		\draw (\x,1pt) -- (\x,-4pt)
			node[anchor=north] {\x};
    	\foreach \y in {0,10,...,180}
     		\draw (1pt,\y) -- (-3pt,\y) 
     			node[anchor=east] {\y}; 
    \draw[->,xshift=0cm] (0,0) -- coordinate (x axis mid) (33,0);
    \draw[->,xshift=0cm] (0,0) -- coordinate (y axis mid)(0,189.2);
	%labels      
	\node[below=0.8cm] at (x axis mid) {num\_threads};
	\node[rotate=90, above=0.8cm] at (y axis mid) {time [s]};

	%plots
	\draw plot[mark=*] file {tif.data};
	\draw plot[mark=square*, mark options={fill=white}] file {tnorm.data};
   
	%legend
	\begin{scope}[shift={(16,70)}] 
	\draw (0,0) -- 
		plot[mark=*] (0.25,0) -- (0.5,0) 
		node[right]{improved};
	\draw[yshift=\baselineskip] (0,10) -- 
		plot[mark=square*, mark options={fill=white}] (0.25,10) -- (0.5,10)
		node[right]{obst};
	\end{scope}
\end{tikzpicture}

Wykres przyśpieszeń:\\
\begin{tikzpicture}[yscale=0.2,xscale=.4, font=\sffamily]
 	%axis
	\draw (0,0) -- coordinate (x axis mid) (33,0);
    	\draw (0,0) -- coordinate (y axis mid) (0,30);
    	%ticks
    	\foreach \x in {0,5,...,32}
     		\draw (\x,1pt) -- (\x,-4pt)
			node[anchor=north] {\x};
    	\foreach \y in {0,5,...,30}
     		\draw (1pt,\y) -- (-3pt,\y) 
     			node[anchor=east] {\y}; 
    \draw[->,xshift=0cm] (0,0) -- coordinate (x axis mid) (33,0);
    \draw[->,xshift=0cm] (0,0) -- coordinate (y axis mid)(0,30);
	%labels      
	\node[below=0.8cm] at (x axis mid) {num\_threads};
	\node[rotate=90, above=0.8cm] at (y axis mid) {speedup};

	%plots
	\draw plot[mark=*] file {s1.data};
	\draw plot[mark=square*, mark options={fill=white}] file {s2.data};
   
	%legend
	\begin{scope}[shift={(16,10)}] 
	\draw (0,0) -- 
		plot[mark=*] (0.25,0) -- (0.5,0) 
		node[right]{s1};
	\draw[yshift=\baselineskip] (0,1) -- 
		plot[mark=square*, mark options={fill=white}] (0.25,1) -- (0.5,1)
		node[right]{s2};
	\end{scope}
\end{tikzpicture}

Poniżej załączam wykresy w rozbiciu na pojedyncze wartości:\\
Wykres OBST:\\
\begin{tikzpicture}[yscale=0.05,xscale=.4, font=\sffamily]
 	%axis
	\draw (0,0) -- coordinate (x axis mid) (33,0);
    	\draw (0,0) -- coordinate (y axis mid) (0,189.2);
    	%ticks
    	\foreach \x in {0,5,...,32}
     		\draw (\x,1pt) -- (\x,-4pt)
			node[anchor=north] {\x};
    	\foreach \y in {0,10,...,180}
     		\draw (1pt,\y) -- (-3pt,\y) 
     			node[anchor=east] {\y}; 
    \draw[->,xshift=0cm] (0,0) -- coordinate (x axis mid) (33,0);
    \draw[->,xshift=0cm] (0,0) -- coordinate (y axis mid)(0,189.2);
	%labels      
	\node[below=0.8cm] at (x axis mid) {num\_threads};
	\node[rotate=90, above=0.8cm] at (y axis mid) {time [s]};

	%plots
	\draw plot[mark=square*, mark options={fill=white}] file {tnorm.data};
   
	%legend
	\begin{scope}[shift={(16,70)}] 
	\draw[yshift=\baselineskip] (0,10) -- 
		plot[mark=square*, mark options={fill=white}] (0.25,10) -- (0.5,10)
		node[right]{obst};
	\end{scope}
\end{tikzpicture}

Wersja z if:\\
\begin{tikzpicture}[yscale=0.05,xscale=.4, font=\sffamily]
 	%axis
	\draw (0,0) -- coordinate (x axis mid) (33,0);
    	\draw (0,0) -- coordinate (y axis mid) (0,189.2);
    	%ticks
    	\foreach \x in {0,5,...,32}
     		\draw (\x,1pt) -- (\x,-4pt)
			node[anchor=north] {\x};
    	\foreach \y in {0,10,...,180}
     		\draw (1pt,\y) -- (-3pt,\y) 
     			node[anchor=east] {\y}; 
    \draw[->,xshift=0cm] (0,0) -- coordinate (x axis mid) (33,0);
    \draw[->,xshift=0cm] (0,0) -- coordinate (y axis mid)(0,189.2);
	%labels      
	\node[below=0.8cm] at (x axis mid) {num\_threads};
	\node[rotate=90, above=0.8cm] at (y axis mid) {time [s]};

	%plots
	\draw plot[mark=*] file {tif.data};
   
	%legend
	\begin{scope}[shift={(16,70)}] 
	\draw (0,0) -- 
		plot[mark=*] (0.25,0) -- (0.5,0) 
		node[right]{improved};
	\end{scope}
\end{tikzpicture}

Wykresy przyśpieszeń (S1):\\
\begin{tikzpicture}[yscale=0.2,xscale=.4, font=\sffamily]
 	%axis
	\draw (0,0) -- coordinate (x axis mid) (33,0);
    	\draw (0,0) -- coordinate (y axis mid) (0,30);
    	%ticks
    	\foreach \x in {0,5,...,32}
     		\draw (\x,1pt) -- (\x,-4pt)
			node[anchor=north] {\x};
    	\foreach \y in {0,5,...,30}
     		\draw (1pt,\y) -- (-3pt,\y) 
     			node[anchor=east] {\y}; 
    \draw[->,xshift=0cm] (0,0) -- coordinate (x axis mid) (33,0);
    \draw[->,xshift=0cm] (0,0) -- coordinate (y axis mid)(0,30);
	%labels      
	\node[below=0.8cm] at (x axis mid) {num\_threads};
	\node[rotate=90, above=0.8cm] at (y axis mid) {speedup};

	%plots
	\draw plot[mark=*] file {s1.data};
   
	%legend
	\begin{scope}[shift={(16,10)}] 
	\draw (0,0) -- 
		plot[mark=*] (0.25,0) -- (0.5,0) 
		node[right]{S1};
	\end{scope}
\end{tikzpicture}

Wykresy przyśpieszeń (S2):\\
\begin{tikzpicture}[yscale=0.2,xscale=.4, font=\sffamily]
 	%axis
	\draw (0,0) -- coordinate (x axis mid) (33,0);
    	\draw (0,0) -- coordinate (y axis mid) (0,30);
    	%ticks
    	\foreach \x in {0,5,...,32}
     		\draw (\x,1pt) -- (\x,-4pt)
			node[anchor=north] {\x};
    	\foreach \y in {0,5,...,30}
     		\draw (1pt,\y) -- (-3pt,\y) 
     			node[anchor=east] {\y}; 
    \draw[->,xshift=0cm] (0,0) -- coordinate (x axis mid) (33,0);
    \draw[->,xshift=0cm] (0,0) -- coordinate (y axis mid)(0,30);
	%labels      
	\node[below=0.8cm] at (x axis mid) {num\_threads};
	\node[rotate=90, above=0.8cm] at (y axis mid) {speedup};

	%plots
	\draw plot[mark=square*, mark options={fill=white}] file {s2.data};
   
	%legend
	\begin{scope}[shift={(16,10)}] 
	\draw (0,1) -- 
		plot[mark=square*, mark options={fill=white}] (0.25,1) -- (0.5,1)
		node[right]{S2};
	\end{scope}
\end{tikzpicture}

Poniżej załączam też tabele dla poszczególnych danych w rozbiciu na pojedyncze wartości:


Wszystkie powyższe wyniki uzyskane zostały na architekturze UCI przy użyciu węzła systemu kolejkowego na klastrze.

\end{document}
